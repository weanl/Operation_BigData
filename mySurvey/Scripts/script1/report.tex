\documentclass[a4paper,UTF8]{article}
\usepackage{ctex}
\usepackage[margin=1.25in]{geometry}
\usepackage{color}
\usepackage{graphicx}
\usepackage{amssymb}
\usepackage{amsmath}
\usepackage{amsthm}
%\usepackage[thmmarks, amsmath, thref]{ntheorem}
\theoremstyle{definition}
\newtheorem*{solution}{Solution}
\newtheorem*{prove}{Proof}
\usepackage{multirow}
\usepackage{url}
\usepackage{enumerate}
\usepackage{algorithm}
\usepackage{algorithmic}
\renewcommand{\algorithmicrequire}{\textbf{Input:}}
\renewcommand{\algorithmicensure}{\textbf{Procedure:}}
\renewcommand\refname{参考文献}

%--

%--
\begin{document}
\title{实验2. 隐马尔科夫模型实践}
\author{MF1733014,高少华,\url{cuggsh@163.com}}
\maketitle

\section*{综述}
	隐马尔科夫模型(Hidden Markov Model, HMM)是一种有向图模型,它用来描述一个含有隐含未知参数
	的马尔可夫过程,在语音识别,自然语言处理等领域有广泛的应用。以下对该模型的简要说明,参考自文献
	\cite{jqxx}的相关章节。\par
	\begin{figure}[!htbp]
		\centering
		\includegraphics[scale=0.40]{hmm.png}
		\caption{隐马尔可夫模型的图结构}
	\end{figure}
	如图1所示,在隐马尔可夫模型中,存在两组变量,其中$\left\{y_{1},y_{2},...,y_{n}\right\}$
	为状态变量,通常是隐藏的,不可被观测的,也称作隐变量。隐变量的取值通常是有N个可能取值的离散空间
	$S=\left\{s_{1},s_{2},...,s_{N}\right\}$。另一组$\left\{x_{1},x_{2},...,x_{n}\right\}$
	为观测变量,可被直接观测到,其取值可以是离散的,也可以是连续的,这里假设其取值也是一个离散空间
	$O=\left\{o_{1},o_{2},...,o_{M}\right\}$。在任意时刻,观测变量的取值仅依赖于隐变量,即
	$x_{t}$由$y_{t}$确定,与其他的隐变量及观测变量无关。同时,${t}$时刻的隐变量$y_{t}$
	仅依赖于${t-1}$时刻的隐变量$y_{t-1}$,即系统下一时刻的状态只与当前时刻的状态有关。则可得所
	有变量的联合概率分布为
	$$P(x_{1},y_{1},...,x_{n},y_{n})=P(y_{1})P(x_{1}|y_{1})\prod_{i=2}^{n}P(y_{i}|y_{i-1})P(x_{i}|y_{i}) \eqno{(1)}$$
	确定一个一个隐马尔可夫模型还需以下三组参数:
	\begin{itemize}
		\item 状态转移矩阵:模型在各个状态间转换的概率,记作$A=[a_{ij}]_{N×N}$,其中$a_{ij}$表示
		在任意时刻从状态$s_{i}$转换到状态$s_{j}$的概率。
		\item 发射矩阵:模型在当前状态获得各个观测值的概率,记作$B=[b_{ij}]_{N×M}$,其中$b_{ij}$表示
		在任意时刻从状态$s_{i}$获取观测值$o_{j}$的概率。
		\item 初始分布:模型在初始时刻各状态出现的概率,记作$\pi=(\pi_{1},\pi_{2},...,\pi_{N})$,
		其中$\pi_{i}$表示模型在初始时刻,状态为$s_{i}$的概率。
	\end{itemize} \par
	隐马尔可夫模型有三个基本问题:
	\begin{itemize}
		\item 已知模型参数$\lambda=[A,B,\pi]$和某一特定的观测序列$\left\{x_{1},x_{2},...,x_{n}\right\}$,
		求最后时刻各个隐状态的概率分布,推测最有可能的观测值$x_{n+1}$。
		\item 已知模型参数$\lambda=[A,B,\pi]$和某一特定的观测序列,求中间时刻各个隐状态的
		概率分布,换言之,就是根据观测序列推测隐藏的模型状态。通常使用 viterbi 算法解决,实验一将实现该算法。
		\item 已知某一特定的观测序列,调整模型参数$\lambda=[A,B,\pi]$,使得该序列出现的概率最大。即,
		训练模型使其能最好地描述观测数据。通常可以使用极大似然法或是 Baum-Welch 算法,来估计模型参数,
		在实验二、三将实现 Baum-Welch 算法的关键步骤。
	\end{itemize}


\section*{实验一. viterbi 算法}
	viterbi 算法是一种动态规划算法,它用于寻找最有可能产生观测序列的维特比路径——隐状态序列,
	特别是在马尔可夫信息源上下文和隐马尔可夫模型中。该算法的伪代码描述如算法1所示\cite{wiki:Viterbi},
	其中输入及输出参数的说明见综述部分,$delta[i,j]$保存在$i$时刻获取到观测状态$s_{j}$的概率的最大值,
	$phi[i,j]$用于记录路径。

	\begin{algorithm}
		\caption{viterbi 算法}
		\begin{algorithmic}[1]
			\REQUIRE $O,S,\pi,X,A,B$
			\ENSURE $Y$
			\FOR{each state $s_{i}$}
			\STATE $delta[i,1] \gets \pi\cdot B_{ix_{1}}$
			\STATE $phi[i,1] \gets 0$
			\ENDFOR
			\FOR{$i \gets 2,3,...,T$}
			\FOR{each state $s_{j}$}
			\STATE $delta[j,i] \gets \max \limits_{k}(delta[k,i-1]\cdot A_{kj} \cdot B{jx_{i}})$
			\STATE $phi[j,i] \gets arg\max \limits_{k}(delta[k,i-1]\cdot A_{kj} \cdot B{jx_{i}})$
			\ENDFOR
			\ENDFOR
			\STATE $Y[T] \gets arg\max \limits_{k}(delta[k,T])$
			\FOR{$i \gets T-1,...,2$}
			\STATE $Y[i] \gets phi[Y[i+1],i+1]$
			\ENDFOR
			\RETURN $Y$
		\end{algorithmic}
	\end{algorithm}


\section*{实验二. Forward 算法}
	前向概率\cite{tjffx}:给定隐马尔可夫模型参数$\lambda = [A, B, \pi]$,到$t$时刻为止,
	观测序列为$o_{1},o_{2},...,o_{t}$且状态为$s_{i}$的概率,记作:
	$$ \alpha_{t}(i) = P(o_{1},o_{2},...,o_{t},y_{t}=s_{i} | \lambda)$$
	可以递归求得前向概率$\alpha_{t}(i)$,Forward 算法描述如下:
	\begin{algorithm}
		\caption{Forward 算法}
		\begin{algorithmic}[1]
			\REQUIRE $O,\lambda=[A, B, \pi]$
			\ENSURE $\alpha$
			\FOR{each state $b_{i}$}
			\STATE $\alpha_{1}(i) \gets \pi_{i}b_{i}(o_{1})$
			\ENDFOR
			\FOR{$t \gets 2,3,...,T$}
			\FOR{each state $b_{i}$}
			\STATE $ \alpha_{t}(i) = [\sum_{j=1}^{N}\alpha_{t-1}(j)a_{ji}]b_{i}(o_{t})$
			\ENDFOR
			\ENDFOR
			\RETURN $\alpha$
		\end{algorithmic}
	\end{algorithm}

\section*{实验三. Backward 算法}
	\begin{algorithm}
		\caption{Backward 算法}
		\begin{algorithmic}[1]
			\REQUIRE $O,\lambda=[A, B, \pi]$
			\ENSURE $\beta$
			\FOR{each state $b_{i}$}
			\STATE $\beta_{T}(i) \gets 1$
			\ENDFOR
			\FOR{$t \gets T-1,T-2,...,1$}
			\FOR{each state $b_{i}$}
			\STATE $ \beta_{t}(i) = [\sum_{j=1}^{N}\beta_{t+1}(j)a_{ji}]b_{i}(o_{t+1})$
			\ENDFOR
			\ENDFOR
			\RETURN $\beta$
		\end{algorithmic}
	\end{algorithm}
	后向概率\cite{tjffx}:给定隐马尔可夫模型参数$\lambda = [A, B, \pi]$,在$t$时刻状态为$s_{i}$的条件下,
	从$t+1$时刻到$T$的部分观测序列为$o_{t+1},o_{t+2},...,o_{T}$的概率,记作:
	$$\beta_{t}(i)=P(o_{t+1},o_{t+2},...,o_{T}|y_{t}=s_{i},\lambda)$$
	可以递归求得前向概率$\beta_{t}(i)$,Backward 算法描述如算法3所示。

\section*{结果}
	程序运行结果如图2所示。
	\begin{figure}[!htbp]
		\centering
		\includegraphics[scale=0.55]{result.png}
		\caption{运行结果}
	\end{figure}

\bibliographystyle{plain}
\bibliography{report}

\end{document}
