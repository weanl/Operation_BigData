\documentclass[a4paper,UTF8]{article}
\usepackage{ctex}
\usepackage[margin=1.25in]{geometry}

\usepackage{url}

\begin{document}
\title{综述. 特征选择方法}
\author{MF1733062 万晨 \url{weanl_jc@163.com}}
\maketitle

\section*{1. 介绍}
  FS在ML中能解决那些问题?已经解决得怎么样了?还有那些问题?feature construction = FS + FE
  特征选择(Feature, Variable and Attribution Selection),是机器学习中feature construction的重要组成部分。在筛选
  原始数据,构造有用的特征集合方面,特征选择不同于特征提取(Feature Extraction):后者会通过线性或非线性的方式从原始数据中
  构造出全新的特征,具有特征学习和表示学习的能力["Representation Learning: A Review and New Perspectives"];特征
  选择通过设计一些简单高效或者精致巧妙的方法,实现从原始特征集合中选出最优的特征子集,能够保持特征对应的原始物理意义。
  所谓最优特征子集,理论上定义为没有信息丢失的最小特征子集,以Markov blanket的形式给出[D. Koller, Toward optimal feature selection][C.Aliferis, Local
  causal and markov blanket induction for causal discovery and feature selection];实际中理论的ground-truth
  很难找,所以经验上一般我们用预测器性能(如分类器的精度)来评估特征子集的选择结果。[Blum and langley,1997,Kohavi and
  John,1997]等在1997年提出了特征选择方面课题研究,当时大多数场景下特征维数不超过40,主要解决的是特征之间的相关性问题。后续
  研究中中涉及的特征维数
  几年在染色体序列分析





\section*{2. 过滤式与包裹式}




\end{document}
